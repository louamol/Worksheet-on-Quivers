\documentclass{worksheetclass}

\usepackage{import}
\import{}{custom_macros.tex}

\title{Quivers}

% DOCUMENT -----------------------------

\begin{document}

\maketitle

\tableofcontents

\section{Quivers representations, path algebras, moduli spaces and quiver varieties}

    \subsection{Quivers, path algebras and relations}

        A quiver $Q$ is a finite directed graph where loops and multiple arrows between edges are allowed. More precisely, it is a piece of combinatorial data $(Q_0,Q_1,t,h)$ such that $Q_0$ is the set of vertice, $Q_1$ the set of arrows and $t,h:Q_1\to Q_0$ are the tail and head maps. A \emph{path} of length $m$ is a trivial composition of arrows $a_1\dots a_m$ such that $h(a_l)=t(a_{l+1})$ for $l=1,\dots,m-1$. It is in particular a \emph{loop} if $h(a_m)=t(a_0)$. We associate to any vertex $i\in Q_0$ a trivial loop $e_i$ going from $i$ to itself. This allows us identify $Q_0$ with the set of paths of lenth $0$. Since $Q_1$ is the set of paths with length $1$ by definition, we can generalize this notation and denote by $Q_m$ be the set of paths of length $m$.
        
        We define the product of paths as
        \begin{equation}
            pq\equiv
            \begin{cases}
                pq,&\quad h(p)=t(q)\\
                0,&\quad\text{else}
            \end{cases}\qquad
            e_ip\equiv
            \begin{cases}
                q,&\quad t(p)=i\\
                0,&\quad\text{else}
            \end{cases}\qquad
            pe_i\equiv
            \begin{cases}
                p,&\quad h(p)=i\\
                0,&\quad\text{else}.
            \end{cases}
        \end{equation}
        We are using the convention that $pq$ means $p$ then $q$. Provided with the formal sum, the set of all paths then forms a ring. Given a field $k$ and an action of $k$ aver this ring we form an algebra, called the \emph{path algebra} of $Q$ and denoted by $kQ$. The addition and multiplication of two elements $a$ and $b$ 
        \begin{equation}
            a=\sum_{\alpha\in Q_m;a_\alpha\in k}\alpha a_\alpha,\qquad b=\sum_{\beta\in Q_m;a_\beta\in k}\beta b_\beta
        \end{equation}
        of $kQ$ are therefore defined as
        \begin{equation}
            a+b=\sum_{\alpha}\alpha(a_\alpha+b_\alpha),\qquad a\cdot b \sum_{\alpha,\beta}\alpha\beta a_\alpha b_\beta.
        \end{equation}
        It is clear that the path algebra is finitely generated if and only if $Q_0$ and $Q_1$ are finite and that it is finite-dimensional if and only if $Q$ has no non-trivial cycles. The $Q_m$'s, the set of paths of length $m$, define a gradation of the path algebra $kQ$:
        \begin{equation}
            kQ=\bigoplus_{m}kQ_m\qquad\text{ with } kQ_m\equiv\bigoplus_{p\in Q_m}pk.
        \end{equation}

        To enforce commutativity of some squares inside a quiver, we can use the notion of relation. A \emph{relation} on a quiver $Q$ is a $k$-linear combination of paths from $Q$. More precisely, is a subsapce of $kQ$ spanned by linear combinations of paths having a common source and common target, and of length at least $2$. A \emph{quiver with relations} is a pair $(Q,I(R))$ where $I(R)\subseteq kQ$ is the (two-sided) ideal of the path algebra generated by $R$. The path algebra of $(Q,I)$ is $kQ/I(R)$. 
        
        \begin{examp*}
            If $Q$ is the $r$-loop, then a relation is a subspace of $kQ = k\langle X_1,\dots,X_r\rangle$ spanned by linear combinations of words of length at least 2. For instance, all the commutators $X_iX_j-X_jX_i$, then the two elements of $kQ$ correspond to the same element in $kQ/I(R)$ if and only their difference is in $I(R)$. In this case, this means that they only differ by inverting the product of two variables in their expression. The resulting algebra is then the ``commutative version'' of $kQ$, which is nothing but than the polynomial algebra $k[X_1,\dots,X_r]$.
        \end{examp*}

    \subsection{Quiver representations}

        A \emph{representation} of a quiver $Q$ over the field $k$ is an assignment to every vertex $i$ os a $k$-vector space $V_i$ and to every arrow $a$ a linear mapping $f_a=V_{t(a)}\to V_{h(a)}$ between the corresponding vector spaces to each arrow. A representation of a quiver with relations $(Q,R)$ has the same definition but with the additional requirement that the linear maps must preserve the relations. \todo{Explain.} Denoting $\alpha_i=\dim V_i$, $\alpha=(\alpha_i)\in\N^{Q_0}_0$ is the \emph{dimension vector} of the representation. Quiver representations are an effective combinatorial tool for organizing linear algebraic data. Not only are they naturally related to many algebraic objects such as quantum groups, Kac-Moody algebras, and cluster algebras, but they have also been studied from the geometric point of view, often serving to bridge the gap between representation theory and algebraic geometry. 
        
        If $V=(V_i,f_a)$ and $W=(W_i,g_a)$ are are two finite-dimensional representation of the same quiver with relations $(Q,R)$, a morphism $\psi$ from $V$ to $W$ is given by specifying , for every vertex $i$, a linear isomorphism $\psi_i:V_i\to W_i$ such that for avery arrow $a$ $\psi_{h(a)}\circ f_a=g_a\circ\psi_{t(a)}$. The \emph{direct sum} $V\oplus W$ of representations can also be defined as $(V\oplus W)_i=V_i\oplus W_i$ with the direct sum of the linear mappings. We then say that a representation is \emph{decomposable} is it is isomorphic to the direct sum of non-zero representations and of \emph{finite-type} (or \emph{finite orbit type}) if it has only finitely many isomorphism classes of indecomposable representations. Or, equivalently, if it has only finitely many isomorphism classes of representations for any prescribed dimension vector. A very important theorem in indecomposable representations is the following:
        \begin{theorem*}[Gabriel]
            A (connected) quiver is of fnite type if and only if its underlying undirected graph is a simply-laced Dynkin diagram.
        \end{theorem*}
        
        To any representation $V=(V_i,f_a)$ of $Q$ we can associate a $k$-vector space
        \begin{equation}
            V=\bigoplus_{i \in Q_0}V_i
        \end{equation}
        equipped with two famillies of linear self-maps: the projections $f_i:V\to V$ ($i\in Q_0$) obtained from the composition $V\hookrightarrow V_i\to V$ of the projections with the inclisions, and tha maps $f_a:V\to V$ ($a\in Q_1$) obtained similarly from the defining maps $f_a=V_{t(a)\to V_{h(a)}}$. One can see that these maps satisfy the relations
        \begin{equation}
            f^2_i=f_i,\qquad f_i\circ f_j=0~(i\neq j),\qquad f_{t(a)}\circ f_a=f_a\circ f_{h(a)}=f_a
        \end{equation}
        and all other products are zero. In this sense, representations a quiver defines an algebra (the algebra of $f_i$ and $f_a$). Now comes the whole point to all this is: this algebra is a representation of the path algebra. To see this, we can use the notation $\rho(e_i)=f_i:V\to V$ and $\rho(a)=f_a:V\to V$ so $\rho:kQ\to \End(V)$ and we can show that is is a $k$-linear morphism of rings. So $\rho$ is a representation of the algebra $kQ$. So a representation of a quiver $Q$ gives a representation of the path algebra. Conversely, given the path algebra $kQ$ of a quiver, then $kQ$ is in particular a vector space over $k$ that yields a familly of vector space $\{V_i=e_i V\}_{i\in Q_0}$ and we have a linear map $f_a:V_{t(a)}\to V_{h(a)}$ for any arrow $a\in Q_1$. We conclude that representations of the path algebra and of the quiver are equivalent. Gabriel's theorem could then be equivalently phrased as: the path algebra of any quiver has finite representations if and only if it is ADE.
        
        Recall also that giving a $k$-linear morphism $A=(k,R)\to\End(V)$ is equivalent to giving a structure of $R$-module to $V$. So from what we discussed above, giving a representation of a quiver is equivalent to giving a module structure to $V$. This can be rephrased using a more categorical language by showing that these constructions extend to functors. It is now clear that the finite-dimensional representations of a quiver with relations form a category, which we denote by $\texttt{fRep}(kQ,R)$. If $(Q,I(R))$ is a quiver with relations and $A=kQ/I$ its path algebra, then the set of finite-dimensional modules of $A$ is also a category that we denote by $\texttt{fdmod}A$. There is a categorical equivalence
        \begin{equation}
            \boxed{\texttt{fRep}(kQ,R)\approx\texttt{fdmod}A.}
        \end{equation}

        The representation of any finite-dimensional alegbra can be described by a quiver with relations $(Q,I)$ such that $I$ contains the power of the ideal generated by the arrows.

    \subsection{Geometric invariant theory}

    \subsection{GIT quotient for quiver representation}

    \subsection{Quiver varities}

    \subsection{Moduli problems}

        Given a collection of algebra-geometric objects it is natural to try to classify these objects up to equivalence. Naively, given a collection $\A$ of such objects and an equivalence relation $\sim$ on $\A$, one may ask whether there exists an algebraic variety $X$ whose points (over the base field $k$) correspond to equivalence classes in $\A/\sim$. This approach is flawed, since there may be many such varieties and some are better than others at retaining the relationships between the objects being classified. If, for example, we are interested in classifying all lines through the origin in the complex plane $\C^2$ up to equality, we can view the corresponding equivalence classes as points of the complex projective line $\P^1_\C$, but we can also see them as points in the disjoint union $\bbA^1_\C\sqcup\{pt\}$ of a line and a point. Ideally, the points of the variety solving a moduli problem should be configured to reflect the relationship between the geometric objects they parametrize.
        
        Thus, in more nuanced approach we try to describe equivalence classes of families of objects of $\A$, rather than just of the objects themselves. That is, we look at pairs $\F,T$, consisting of a variety $T$ and a family $\pi:\F\to X$ of objects of $\A$ (the fibers $\pi^{-1}(t)$ are objects of $\A$) subject to some additional conditions (e.g. that $\pi$ is flat). Moreover, if the collection of families over a variety $T$ is denoted by $\A_T$, then for any morphism $f:S\to T$, there should be a pullback operation assigning to any family $\F\in \A_T$ a family $f^*\F\in\A_S$. Now, we can extend our moduli problem by introducing an equivalence relation $\sim_T$ on $A_T$ that is compatible with pullback and give us the starting equivalence relation $\sim$ when $T$ is $\text{Spec}k$.
        
        The solution to such an extended moduli problem, called a \emph{fine moduli space}, consists of a variety $X$ whose $k$-points classify equivalence classes in $\A/\sim$, together with a \emph{universal family} $\mathcal{U}\in\A_X$, which describes how these equivalence classes relate to each other. More specifically, any family $\F\in\A_T$ over a variety $T$ is equivalent to the pullback $f^*\mathcal{U}$ along a unique morphism $f:T\to X$. In the special case that $T=\text{Spec}k$, we obtain that the $k$-points of $X$ are in bijection with equivalence classes in $\A$. Furthermore, it turns out that the fine moduli space is
        unique up to isomorphism. 
        
        Considering once again the example of lines through the origin in $\C^2$, we see that a family of such lines over a variety $T$ may be thought of as a line subbundle $\mathcal{L}\subset\C^2\times T\to T$ of the trivial rank $2$ vector bundle over $T$. The equivalence relation becomes an isomorphism of line subbundles of $\C^2\times T$. The solution to the corresponding moduli problem consists of the complex projective line $\P^1_\C$ together with the tautological line bundle $\mathcal{O}_{\P^1_\C}(-1)$. Indeed, if $\mathcal{L}\subset\C^2\times T$ is a subbundle, then its dual $\mathcal{L}^\vee$ is generated by global sections (specifically, the images of the standard global sections with respect to the surjection   $(\C^2)^\vee\times T\twoheadrightarrow\mathcal{L}^\vee)$. This defines a unique morphism $f:T\to\P^1_\C$ such that $\mathcal{L}\simeq f^*\mathcal{O}(-1)$.

        Unfortunately, it is often the case that, for a given class of objects $\A$, a fine modduli space either does not exist or requires us to place restrictions on the kinds of objects in $\A$ we wish to classify. In order to avoid this, we can forget the universal family and look for a nice enough variety with points in bijection with equivalence classes in $\A/\sim$. Alternatively, we can allow for the solution of our moduli problem to no longer be a variety (or even a scheme). The result is a more complicated object called a \emph{stack}.

\printbibliography

\end{document}